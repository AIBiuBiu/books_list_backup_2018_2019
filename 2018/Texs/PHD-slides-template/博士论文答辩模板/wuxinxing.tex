\documentclass[												
size=10pt,															
paper=screen,														
mode=present,														
orient=landscape,
clock=true,															
style=chenandwu		
]{powerdot}
\usepackage{graphicx}						
\usepackage{fancyvrb}									
\usepackage{CJK}
\usepackage{overpic}
\usepackage{color}
\usepackage{multimedia} %插入电影文件
\usepackage{wrapfig}
\usepackage{picins}
\usepackage{amsfonts,amsmath,amssymb,amsthm}                                                 
\usepackage[mathscr]{eucal}
\newcommand{\myfontsizea}{\fontsize{80pt}{80pt}\selectfont}
\newcommand{\myfontsizeb}{\fontsize{5.45pt}{5.40pt}\selectfont}
%============================================================================================
\definecolor{mycolor}{rgb}{0.7,0,0}
\pdsetup{trans=Replace
}
%============================================================================================
%%%%%%%%%%%%%%%%%%%%%%%%%%%%%%%%%%%%%%%%%%%%%%%%%%%%%%%%%%%%%%%%%%%%%%%%%%%%%%%%%
% Slide 1                                                                       %
%%%%%%%%%%%%%%%%%%%%%%%%%%%%%%%%%%%%%%%%%%%%%%%%%%%%%%%%%%%%%%%%%%%%%%%%%%%%%%%%%
\begin{document}
\begin{CJK*}{GBK}{song}
\title{\LARGE{\textbf{基于语言的软件可信性度量}\\\textbf{理论及其应用}} \\[1.5em]
}		
\author{答辩人: 吴新星 \\[0.6em]
导师: 陈仪香\, 教授\\[1.8em]
\small{软件学院} \\[0.6em]
\small{2011年6月1日}\\[0.9em]
\date{}
}	
\maketitle												
%%%%%%%%%%%%%%%%%%%%%%%%%%%%%%%%%%%%%%%%%%%%%%%%%%%%%%%%%%%%%%%%%%%%%%%%%%%%%%%%%
% Slide 2                                                                       %
%%%%%%%%%%%%%%%%%%%%%%%%%%%%%%%%%%%%%%%%%%%%%%%%%%%%%%%%%%%%%%%%%%%%%%%%%%%%%%%%%
\begin{slide}{提纲}	
\tableofcontents[content=sections]
\end{slide}															
%%%%%%%%%%%%%%%%%%%%%%%%%%%%%%%%%%%%%%%%%%%%%%%%%%%%%%%%%%%%%%%%%%%%%%%%%%%%%%%%%
% Slide 3                                                                       %
%%%%%%%%%%%%%%%%%%%%%%%%%%%%%%%%%%%%%%%%%%%%%%%%%%%%%%%%%%%%%%%%%%%%%%%%%%%%%%%%%
\section{1.内容概要}	
\begin{slide}{背景介绍}
\begin{itemize}
\item{软件: 多次修改、多种语言、多人合作、规模庞大\textbf{\ldots\ldots}{\footnotemark[1]}}

\end{itemize}
{\footnotetext[1]{{\tiny{陈仪香. 信息技术当前的研究热点[J]. 模糊系统与数学, 2010, 24: 1-3(增刊)}}}}
\end{slide}
%%%%%%%%%%%%%%%%%%%%%%%%%%%%%%%%%%%%%%%%%%%%%%%%%%%%%%%%%%%%%%%%%%%%%%%%%%%%%%%%%
% Slide 4                                                                       %
%%%%%%%%%%%%%%%%%%%%%%%%%%%%%%%%%%%%%%%%%%%%%%%%%%%%%%%%%%%%%%%%%%%%%%%%%%%%%%%%%
\begin{slide}[toc=,bm=]{背景介绍}
\begin{itemize}
\item 这座城市的中央计算机告诉你的? R2D2, 你不该相信一台陌生的计算机!
\end{itemize}
\end{slide}
%%%%%%%%%%%%%%%%%%%%%%%%%%%%%%%%%%%%%%%%%%%%%%%%%%%%%%%%%%%%%%%%%%%%%%%%%%%%%%%%%
% Slide 5                                                                       %
%%%%%%%%%%%%%%%%%%%%%%%%%%%%%%%%%%%%%%%%%%%%%%%%%%%%%%%%%%%%%%%%%%%%%%%%%%%%%%%%%
\section{2.论文的工作}	
\section{\quad (1) 理论}
\section{\qquad 随机-混成进程代数可信性量化 (第三章)}
\begin{slide}{随机-混成进程代数可信性量化}
随机-混成进程代数可信性量化
\end{slide}
%%%%%%%%%%%%%%%%%%%%%%%%%%%%%%%%%%%%%%%%%%%%%%%%%%%%%%%%%%%%%%%%%%%%%%%%%%%%%%%%%
% Slide 6                                                                       %
%%%%%%%%%%%%%%%%%%%%%%%%%%%%%%%%%%%%%%%%%%%%%%%%%%%%%%%%%%%%%%%%%%%%%%%%%%%%%%%%%
\section{3.总结和进一步工作}
\begin{slide}{本文贡献}
基于语言建立软件可信性度量\textcolor[rgb]{0.70,0.00,0.00}{理论}, \textcolor[rgb]{0.70,0.00,0.00}{应用}到基于结构化程序设计语言的软件、构件的软件和BPEL的Web服务可信性度量的研究, 使用Ruby/Tk 开发可信性度量\textcolor[rgb]{0.70,0.00,0.00}{工具}.
\begin{enumerate}
\item<2-5> 随机-混成进程代数可信性量化

将此理论方法应用到基于结构化程序设计语言的软件和基于BPEL语言的 Web 服务可信性度量的研究
\item<3-5> 概率拟 Hoare 逻辑

将此理论应用到基于构件的软件可信性度量的研究
\item<4-5> Web服务降级替换可信性量化理论
\item<5> 软件可信性度量可视化工具
\end{enumerate}
\end{slide}
%%%%%%%%%%%%%%%%%%%%%%%%%%%%%%%%%%%%%%%%%%%%%%%%%%%%%%%%%%%%%%%%%%%%%%%%%%%%%%%%%
% Slide 6                                                                       %
%%%%%%%%%%%%%%%%%%%%%%%%%%%%%%%%%%%%%%%%%%%%%%%%%%%%%%%%%%%%%%%%%%%%%%%%%%%%%%%%%
\begin{slide}[method=direct]{进一步工作}
\begin{itemize}
\item 有关度量理论和模型的原子构造可信度只是理论上的, 接下来我们将进一步考虑通过具体的方法在实际中来估计原子构造的可信度.

\item 丰富可信性度量规则, 进一步考虑对软件可信性的动态分析和度量以及测试其作为可信性度量工具自身的可信性.
\item \ldots\ldots
\end{itemize}
\end{slide}
%%%%%%%%%%%%%%%%%%%%%%%%%%%%%%%%%%%%%%%%%%%%%%%%%%%%%%%%%%%%%%%%%%%%%%%%%%%%%%%%%
% Slide 7                                                                      %
%%%%%%%%%%%%%%%%%%%%%%%%%%%%%%%%%%%%%%%%%%%%%%%%%%%%%%%%%%%%%%%%%%%%%%%%%%%%%%%%%
\begin{slide}[method=direct]{已发表的工作}

{\textcircled{1} XXXXXXXXXXXXXXXXXXXX}
\end{slide}
%%%%%%%%%%%%%%%%%%%%%%%%%%%%%%%%%%%%%%%%%%%%%%%%%%%%%%%%%%%%%%%%%%%%%%%%%%%%%%%%%
% Slide 8                                                                      %
%%%%%%%%%%%%%%%%%%%%%%%%%%%%%%%%%%%%%%%%%%%%%%%%%%%%%%%%%%%%%%%%%%%%%%%%%%%%%%%%%
\begin{slide}[method=direct]{参与的科研项目}
{\small{
\begin{itemize}
\item 参与国家 (863) 高技术研究发展计划项目(2007AA01Z189)---软件可信性度量模型的研究 \quad 2007.07—2009.12
\end{itemize}
}}
\end{slide}
%%%%%%%%%%%%%%%%%%%%%%%%%%%%%%%%%%%%%%%%%%%%%%%%%%%%%%%%%%%%%%%%%%%%%%%%%%%%%%%%%
% Slide 9                                                                       %
%%%%%%%%%%%%%%%%%%%%%%%%%%%%%%%%%%%%%%%%%%%%%%%%%%%%%%%%%%%%%%%%%%%%%%%%%%%%%%%%%
\section{4.致谢}								
\begin{slide}{致谢}
\begin{overpic}[scale=0.4]{ppt4111.eps}

\onslide{1}{\put(2.0,55){\color{mycolor}\textbf{\huge{感谢我的导师陈仪香教授!}}}}

\onslide*{1}{\put(2.0,35){\color{mycolor}\textbf{\huge{感谢FICS小组的老师和同学们!}}}}

\end{overpic}
\end{slide}
%%%%%%%%%%%%%%%%%%%%%%%%%%%%%%%%%%%%%%%%%%%%%%%%%%%%%%%%%%%%%%%%%%%%%%%%%%%%%%%%%
% Slide 10                                                                      %
%%%%%%%%%%%%%%%%%%%%%%%%%%%%%%%%%%%%%%%%%%%%%%%%%%%%%%%%%%%%%%%%%%%%%%%%%%%%%%%%%
\pdsetup{trans=Fade}
\begin{emptyslide}[toc=,bm=]{}
\vspace{11.5em}
\begin{center}
{\Huge{\textcolor[rgb]{0.70,0.00,0.00}{\textbf{谢谢大家!}}}} 
\end{center}

\end{emptyslide}

\end{CJK*}
\end{document}
\endinput

