% float.tex
%
% Aetf <aetf@unlimitedcodeworks.xyz>
% Copyright 2016 Aetf <aetf@unlimitedcodeworks.xyz>
%
% multiple1902 <multiple1902@gmail.com>
% Copyright 2011~2012, multiple1902 (Weisi Dai)
%
% Project Home: https://github.com/Aetf/xjtuthesis
%
% It is strongly recommended that you read documentations located at
%   https://github.com/Aetf/xjtuthesis/wiki
% in advance of your compilation if you have not read them before.
%
% This work may be distributed and/or modified under the
% conditions of the LaTeX Project Public License, either version 1.3
% of this license or (at your option) any later version.
% The latest version of this license is in
%   http://www.latex-project.org/lppl.txt
% and version 1.3 or later is part of all distributions of LaTeX
% version 2005/12/01 or later.
%
% This work has the LPPL maintenance status `maintained'.
%
% The Current Maintainer of this work is Aetf.
%
\chapter{浮动格式}

    金溪民方仲永,世隶耕。仲永生五年,未尝识书具,忽啼求之。父异焉,借旁近与之,即书诗四句,并自为其名。其诗以养父母、收族为意,传一乡秀才观之。自是指物作诗立就,其文理皆有可观者。邑人奇之,稍稍宾客其父,或以钱币乞之。父利其然也,日扳仲永环谒于邑人,不使学。

    余闻之也久。明道中,从先人还家,于舅家见之,十二三矣。令作诗,不能称前时之闻。又七年,还自扬州,复到舅家问焉。曰:“泯然众人矣。”

    王子曰:仲永之通悟,受之天也。其受之天也,贤于才人远矣。卒之为众人,则其受于人者不至也。彼其受之天也,如此其贤也,不受之人,且为众人;今夫不受之天,固众人,又不受之人,得为众人而已耶?

    \section{图片}

        \begin{figure}[h!]
          \centering
          \includegraphics[width=6.67cm]{XJTU.pdf}
          \caption{西安交通大学}
          \label{fig:xjtu}
        \end{figure}

        \begin{figure}[h!]
          \begin{minipage}{0.45\textwidth}
              \centering
              \includegraphics[width=6.67cm]{XJTU.pdf}
              \caption{西安交通大学}
              \label{fig:xjtu-left}
          \end{minipage}
          \begin{minipage}{0.45\textwidth}
              \centering
              \includegraphics[width=6.67cm]{XJTU.pdf}
              \caption{西安交通大学}
              \label{fig:xjtu-right}
          \end{minipage}
        \end{figure}
          
        \begin{figure}[h!]
          \centering
          \subfloat[果毅力行]{
              \includegraphics[width=6.67cm]{XJTU.pdf}
              \label{fig:xjtu-sub-left}}
          \subfloat[忠恕任事]{
              \includegraphics[width=6.67cm]{XJTU.pdf}
              \label{fig:xjtu-sub-right}}
          \caption{子图}
        \end{figure}
          

    \section{表格}

        \begin{table}[h!]
          \centering
          \caption{一个简单的表格}
          \label{tab:simple}
          \wuhao
          \begin{tabularx}{\linewidth}{XXXXX} \toprule 
                & 一月 & 二月 & 三月 & 合计 \\ \midrule
           东部 &    7 &    7 &    5 &   19 \\ 
           西部 &    6 &    4 &    7 &   17 \\ 
           南部 &    8 &    7 &    9 &   24 \\ 
       \bf 合计 &   21 &   18 &   21 &   60 \\ \bottomrule
          \end{tabularx}
        \end{table}


        \begin{table}[h!]
          %\begin{minipage}{\textwidth}
          \begin{threeparttable}[h]
            \centering
            \caption{包含脚注的表格}
            \label{tab:with-footnote}
            \wuhao
            \begin{tabularx}{\linewidth}{XXXXX} \toprule 
                  & 一月 & 二月 & 三月 & 合计 \\ \midrule
                  东部 &    7\tnote{1}
                                &    7 &    5 &   19 \\ 
             西部 &    6 &    4 &    7 &   17 \\ 
             南部 &    8 &    7 &    9 &   24 \\ 
             \bf 合计\tnote{2}
                  &   21 &   18 &   21 &   60 \\ \bottomrule
            \end{tabularx}
          %\end{minipage}
          \begin{tablenotes}
          \item[1] 数据来自Word 97.
          \item[2] Computed by \textsl{Mathematica} 8.
          \end{tablenotes}
          \end{threeparttable}
        \end{table}

        \begin{table}[h!]
          \centering
          \caption{稍微复杂一点的表格}
          \label{tab:complex}
          \wuhao
          \begin{tabularx}{\linewidth}{XXXXX} \toprule 
                & \multicolumn{3}{c}{这是一句废话} &  \\ \cmidrule{2-4}
                & 一月 & 二月 & 三月 & 合计 \\ \midrule
           东部 &    7 &    7 &    5 &   19 \\ 
           西部 &    6 &    4 &    7 &   17 \\ 
           南部 &    8 &    7 &    9 &   24 \\ 
       \bf 合计 &   21 &   18 &   21 &   60 \\ \bottomrule
          \end{tabularx}
        \end{table}

        我制作了一个简单的表格(表\ref{tab:simple})。

