% formulae.tex
%
% Aetf <aetf@unlimitedcodeworks.xyz>
% Copyright 2016 Aetf <aetf@unlimitedcodeworks.xyz>
%
% multiple1902 <multiple1902@gmail.com>
% Copyright 2011~2012, multiple1902 (Weisi Dai)
%
% Project Home: https://github.com/Aetf/xjtuthesis
%
% It is strongly recommended that you read documentations located at
%   https://github.com/Aetf/xjtuthesis/wiki
% in advance of your compilation if you have not read them before.
%
% This work may be distributed and/or modified under the
% conditions of the LaTeX Project Public License, either version 1.3
% of this license or (at your option) any later version.
% The latest version of this license is in
%   http://www.latex-project.org/lppl.txt
% and version 1.3 or later is part of all distributions of LaTeX
% version 2005/12/01 or later.
%
% This work has the LPPL maintenance status `maintained'.
%
% The Current Maintainer of this work is Aetf.
%
\chapter{公式环境}

    \begin{axiom}
        \rm 两点间直线段距离最短。  
        \begin{align}
            x&\equiv y+1\pmod{m^2}\\
            x&\equiv y+1\mod{m^2}\\
            x&\equiv y+1\pod{m^2}
        \end{align}
    \end{axiom}

    \begin{remark}
    \rm 对齐的公式示例,它还同时演示了标号。
    \begin{align}
    \begin{split} 
    \varphi(x,z)
    &=z-\gamma_{10}x-\gamma_{mn}x^mz^n\\
    &=z-Mr^{-1}x-Mr^{-(m+n)}x^mz^n
    \end{split} \notag \\
    \noindent\zeta^1&=(\xi^1)^2,\\
    \zeta^1 &=\xi^0\xi^1,\\
    \zeta^2 &=(\xi^1)^2,
    \end{align}
    \end{remark}

    \begin{theorem}
      \rm 对于直角三角形$ABC$, 若$a<c$且$b<c$, 则有
        \begin{equation}
          a^2+b^2=c^2
        \end{equation}
    \end{theorem}


    \begin{exercise}
          \rm 请列出温家宝的所有影视作品。
    \end{exercise}
        
    贝叶斯公式如式~(\ref{equ:chap1:bayes}),其中 $p(y|\mathbf{x})$ 为后验;
    $p(\mathbf{x})$ 为先验;分母 $p(\mathbf{x})$ 为归一化因子。
    \begin{equation}
        \label{equ:chap1:bayes}
        p(y|\mathbf{x}) = \frac{p(\mathbf{x},y)}{p(\mathbf{x})}=
        \frac{p(\mathbf{x}|y)p(y)}{p(\mathbf{x})} 
    \end{equation}
