% !TEX program = xelatex
% This is my resume
% Chinese translation
% by ice1000

\documentclass{resume}

\usepackage{lastpage}
\usepackage{fancyhdr}
\usepackage{linespacing_fix} % disable extra space before next section
\usepackage[fallback]{xeCJK}

%% \setmainfont[]{SimSun}
%% \setCJKfallbackfamilyfont{rm}{HAN NOM B}
\setCJKmainfont[BoldFont=SimHei,ItalicFont=KaiTi_GB2312]{SimSun}
%% \renewcommand{\thepage}{\Chinese{page}}

\begin{document}
\pagestyle{fancy}
\fancyhf{}
\renewcommand\headrulewidth{0pt}
\cfoot{\thepage\ of \pageref{LastPage}}

\name{张寅森}

\basicInfo{
  \email{ice1000kotlin@foxmail.com} \textperiodcentered\ 
  \phone{(+86) 180-8192-5082} \textperiodcentered\ 
  \github[ice1000]{https://github.com/ice1000}
  % \linkedin[billryan8]{https://www.linkedin.com/in/billryan8}
}

\section{\faGraduationCap\ 教育经历}
\datedsubsection{\textbf{宾夕法尼亚州州立大学}, 美国}{2018.8 -- 现在}
  专业:计算机科学,预计毕业日期:2022.6

\section{\faUsers\ 工作经历}
\datedsubsection{\textbf{前海源伞}, 深圳, 中国}{2018.2 -- 2018.7}
\role{实习}{编译器前端,IDE 插件开发}
\begin{itemize}
  \item 负责 pinpoint 分析器的 IntelliJ/CLion/Eclipse 工具集成,协助开发 SonarQube 插件
  \item 编写了一个多线程的跨 Java/Kotlin 的源代码索引工具,索引 hadoop 仅需 4 分钟

  \item 学到了很多 Linux 编程和 clang 源码相关的知识
\end{itemize}

\datedsubsection{\textbf{PingCAP}, 远程}{2018.8 -- 现在}
\role{实习}{TiKV 组}
\begin{itemize}
  \item 优化
    \href{https://docs.rs/crate/grpcio} {grpcio}
    (gRPC 的 Rust 绑定)的内存交互性能
\end{itemize}

\section{\faGithubAlt\ 个人项目}
% \datedsubsection{\textbf{DevKt}}{\url{https://icela.github.io/}}
% % \role{Kotlin, C\#, Racket, Ruby}{发起者和 Kotlin/C\#/Ruby 版本的主维护者}
% 一个跨语言、易使用的游戏引擎系列。
% \begin{itemize}
%   % \item 易于使用,仅需实现生命周期方法,然后调用非常方便的 API 。
%   % \item 易于安装,基于各语言内置的 GUI 框架。
%   \item 基于生命周期方法和工具 API ,并基于各语言内置的 GUI 框架,易于安装。
%   \item 使用 GitHub 的 issue 和 milestone 功能作为任务和版本管理工具,有完善的改动记录和文档。
%   \item 提供一个基于 JavaFX 的拖拽式可视化设计器,所见即所得,可以生成各种语言的代码。
% \end{itemize}

\datedsubsection{\textbf{DevKt}}{\url{https://github.com/ice1000/dev-kt}}
跨平台轻量级代码编辑器兼 Kotlin IDE。
\begin{itemize}
  \item 内置 Java/Kotlin 的高亮、补全,其他语言可以借助插件(可移植自 JetBrains IDE)做到同样的支持。 \\
    对 Kotlin 有额外的编译运行支持。
  \item 架构灵活,编辑器上层逻辑和 UI 框架彻底解藕,易于向其他 UI 框架移植。
  \item 提供细粒度的高亮颜色和快捷键设置,设置可以热更新。
\end{itemize}

\datedsubsection{\textbf{Lice 语言}}{\url{https://github.com/lice-lang/lice}}
% \role{Kotlin, Java}{发起者和主维护者}
高度可扩展的解释型程序语言,运行在 JVM 上。
\begin{itemize}
  \item 支持 lambda 和 惰性求值 (call by need) / 正则序求值 (call by name) / 严格求值 (call by value)。
  \item 运行速度约为 Java (Hotspot 8u151) 的二十分之一,提供 Java 交互和脚本引擎支持。
  \item 提供支持 GHCi 风格的代码补全、彩色输出的命令行交互式解释器和支持基于语义的高亮、补全、重命名、 \\
    定义跳转、求值替换、快速修复等功能的 JetBrains IDE 插件。
\end{itemize}

\datedsubsection{\textbf{Julia-IntelliJ}}{\url{https://github.com/ice1000/julia-intellij}}
JetBrains IDE 的 Julia 插件,支持全线 JetBrains 产品。
\begin{itemize}
  \item 支持基于语义的高亮、错误检查、快速修复、定义跳转、参数提示、补全、针对 Unicode 字符的特殊输入。
  \item 集成 Markdown 插件高亮文档字符串,提供 REPL、SciView(展示 Plot 库的输出) 支持
\end{itemize}

% Reference Test
%\datedsubsection{\textbf{Paper Title\cite{zaharia2012resilient}}}{May. 2015}
%An xxx optimized for xxx\cite{verma2015large}
%\begin{itemize}
%  \item main contribution
%\end{itemize}

%% \section{\faHeartO\ 成就}
%% \datedline{}{Aug. 2017}

\section{\faCogs\ 技能}
\begin{itemize}[parsep=0.25ex]
  \item \textbf{编程语言}:
    \textbf{泛语言开发者}(编程不受特定语言限制),
    且尤其熟悉 Java/Kotlin/Rust/C\#/Agda/Haskell,
    较为熟悉 Dart/C++/F\#/F$\star$ (均不分先后)

  % compiler theories
  \item \textbf{编译原理}:
    熟练使用各种 Parser Generator,了解一些依赖类型系统实现细节
    (如 Luo's UTT,Martin-Löf 类型论)

  % language Kotlin
  \item \textbf{Kotlin/Java}:
    \textbf{2 年开发经验},
    \textbf{4} 个项目被
    \href{https://kotlin.link/?q=ice} {Awesome Kotlin}
    收录,熟悉 JNI 编程、Gradle 构建工具,
    有使用 Kotlin 编译器分析 Java 代码的经验

  \item \textbf{形式验证}:
    掌握依赖类型、依赖模式匹配、高阶归纳类型,正在学习同伦类型论,
    熟悉 Idirs, Agda (\textbf{1 年}使用经验,开发组成员之一)
    和 F$\star$ 编程语言
    \subitem 读过代码的项目: Agda, Idris, miniagda

  \item \textbf{JetBrains MPS}:
    \textbf{1 年开发经验},
    理解\textbf{面向语言编程}的概念和应用

  \item \textbf{IDE 工具开发}:
    \textbf{2 年开发经验},
    熟悉 IntelliJ 平台的基础设施和整体架构(开发了
      \href{https://plugins.jetbrains.com/plugin/10413-julia}
           {Julia 插件} 和很多其他插件),
    也了解 Eclipse/SonarQube 的插件开发

  % platforms I am familiar with  
  \item \textbf{移动开发}:
    \textbf{2 年开发经验},
    Android (Java, Kotlin), Fuchsia (Flutter)

  \item \textbf{开发工具}:
    能适应任何编辑器/操作系统,平常在 Ubuntu 下使用 JetBrains IDE、Emacs,
    有使用 YouTrack、Jira、GitHub、BitBucket、Coding.net、Tower 等团队协作工具的经验
\end{itemize}

% \section{\faHeartO\ Honors and Awards}
% \datedline{\textit{\nth{1} Prize}, Award on xxx }{Jun. 2013}
% \datedline{Other awards}{2015}

\section{\faInfo\ 其他}
\begin{itemize}[parsep=0.25ex]
  \item 博客: \url{https://ice1000.org/} 部分翻译为英语
  \item Literate Agda 博客: \url{https://ice1000.org/lagda/} 使用了我改进过的 Agda 文学编程模式
  \item Bintray: \url{https://bintray.com/ice1000} 发布了一些好用的 JVM 库
  \item IntelliJ 插件开发者主页: \url{https://plugins.jetbrains.com/author/10a216dd-c558-4aaf-aa8a-723f431452fb}
  \item 我写的一些关于形式验证的书: \url{https://github.com/ice1000/Books}
  \item 开源贡献: \url{http://ice1000.org/opensource-contributions/} \\
    向 \textit{Microsoft, JetBrains, Ruby, Dropbox, PingCAP, TiKV} 等组织,
    \textit{agda, imgui, shields.io, intellij-solidity, IntelliJ-EmmyLua,
      intellij-haskell} 等项目提交过功能性的 pull request
  \item StackOverflow: \url{http://stackoverflow.com/users/7083401/}
    3000+ 声誉,同时也在 \href{https://stackexchange.com/users/9532102/} {其他 StackExchange 站点} 活跃
  \item 语言: English - 熟练 (托福 100),汉语 - 母语水平
  % https://raw.githubusercontent.com/ice1000/resume/master/resume-cn.pdf
  \item 获取此简历的最新版本: \url{https://tinyurl.com/ya4urea8}
  \item 在
    \href{https://www.codewars.com/users/ice1000} {CodeWars}
    上,以 Haskell 为主,达到
    \textbf{1 kyu},
    全站排名 \#37
\end{itemize}

%% Reference
%\newpage
%\bibliographystyle{IEEETran}
%\bibliography{mycite}
\end{document}
